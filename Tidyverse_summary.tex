% Options for packages loaded elsewhere
\PassOptionsToPackage{unicode}{hyperref}
\PassOptionsToPackage{hyphens}{url}
%
\documentclass[
]{article}
\usepackage{lmodern}
\usepackage{amsmath}
\usepackage{ifxetex,ifluatex}
\ifnum 0\ifxetex 1\fi\ifluatex 1\fi=0 % if pdftex
  \usepackage[T1]{fontenc}
  \usepackage[utf8]{inputenc}
  \usepackage{textcomp} % provide euro and other symbols
  \usepackage{amssymb}
\else % if luatex or xetex
  \usepackage{unicode-math}
  \defaultfontfeatures{Scale=MatchLowercase}
  \defaultfontfeatures[\rmfamily]{Ligatures=TeX,Scale=1}
\fi
% Use upquote if available, for straight quotes in verbatim environments
\IfFileExists{upquote.sty}{\usepackage{upquote}}{}
\IfFileExists{microtype.sty}{% use microtype if available
  \usepackage[]{microtype}
  \UseMicrotypeSet[protrusion]{basicmath} % disable protrusion for tt fonts
}{}
\makeatletter
\@ifundefined{KOMAClassName}{% if non-KOMA class
  \IfFileExists{parskip.sty}{%
    \usepackage{parskip}
  }{% else
    \setlength{\parindent}{0pt}
    \setlength{\parskip}{6pt plus 2pt minus 1pt}}
}{% if KOMA class
  \KOMAoptions{parskip=half}}
\makeatother
\usepackage{xcolor}
\IfFileExists{xurl.sty}{\usepackage{xurl}}{} % add URL line breaks if available
\IfFileExists{bookmark.sty}{\usepackage{bookmark}}{\usepackage{hyperref}}
\hypersetup{
  pdftitle={젊은 임상 연구자들을 위한 필수 R},
  pdfauthor={Jonggi Choi},
  hidelinks,
  pdfcreator={LaTeX via pandoc}}
\urlstyle{same} % disable monospaced font for URLs
\usepackage[margin=1in]{geometry}
\usepackage{color}
\usepackage{fancyvrb}
\newcommand{\VerbBar}{|}
\newcommand{\VERB}{\Verb[commandchars=\\\{\}]}
\DefineVerbatimEnvironment{Highlighting}{Verbatim}{commandchars=\\\{\}}
% Add ',fontsize=\small' for more characters per line
\usepackage{framed}
\definecolor{shadecolor}{RGB}{248,248,248}
\newenvironment{Shaded}{\begin{snugshade}}{\end{snugshade}}
\newcommand{\AlertTok}[1]{\textcolor[rgb]{0.94,0.16,0.16}{#1}}
\newcommand{\AnnotationTok}[1]{\textcolor[rgb]{0.56,0.35,0.01}{\textbf{\textit{#1}}}}
\newcommand{\AttributeTok}[1]{\textcolor[rgb]{0.77,0.63,0.00}{#1}}
\newcommand{\BaseNTok}[1]{\textcolor[rgb]{0.00,0.00,0.81}{#1}}
\newcommand{\BuiltInTok}[1]{#1}
\newcommand{\CharTok}[1]{\textcolor[rgb]{0.31,0.60,0.02}{#1}}
\newcommand{\CommentTok}[1]{\textcolor[rgb]{0.56,0.35,0.01}{\textit{#1}}}
\newcommand{\CommentVarTok}[1]{\textcolor[rgb]{0.56,0.35,0.01}{\textbf{\textit{#1}}}}
\newcommand{\ConstantTok}[1]{\textcolor[rgb]{0.00,0.00,0.00}{#1}}
\newcommand{\ControlFlowTok}[1]{\textcolor[rgb]{0.13,0.29,0.53}{\textbf{#1}}}
\newcommand{\DataTypeTok}[1]{\textcolor[rgb]{0.13,0.29,0.53}{#1}}
\newcommand{\DecValTok}[1]{\textcolor[rgb]{0.00,0.00,0.81}{#1}}
\newcommand{\DocumentationTok}[1]{\textcolor[rgb]{0.56,0.35,0.01}{\textbf{\textit{#1}}}}
\newcommand{\ErrorTok}[1]{\textcolor[rgb]{0.64,0.00,0.00}{\textbf{#1}}}
\newcommand{\ExtensionTok}[1]{#1}
\newcommand{\FloatTok}[1]{\textcolor[rgb]{0.00,0.00,0.81}{#1}}
\newcommand{\FunctionTok}[1]{\textcolor[rgb]{0.00,0.00,0.00}{#1}}
\newcommand{\ImportTok}[1]{#1}
\newcommand{\InformationTok}[1]{\textcolor[rgb]{0.56,0.35,0.01}{\textbf{\textit{#1}}}}
\newcommand{\KeywordTok}[1]{\textcolor[rgb]{0.13,0.29,0.53}{\textbf{#1}}}
\newcommand{\NormalTok}[1]{#1}
\newcommand{\OperatorTok}[1]{\textcolor[rgb]{0.81,0.36,0.00}{\textbf{#1}}}
\newcommand{\OtherTok}[1]{\textcolor[rgb]{0.56,0.35,0.01}{#1}}
\newcommand{\PreprocessorTok}[1]{\textcolor[rgb]{0.56,0.35,0.01}{\textit{#1}}}
\newcommand{\RegionMarkerTok}[1]{#1}
\newcommand{\SpecialCharTok}[1]{\textcolor[rgb]{0.00,0.00,0.00}{#1}}
\newcommand{\SpecialStringTok}[1]{\textcolor[rgb]{0.31,0.60,0.02}{#1}}
\newcommand{\StringTok}[1]{\textcolor[rgb]{0.31,0.60,0.02}{#1}}
\newcommand{\VariableTok}[1]{\textcolor[rgb]{0.00,0.00,0.00}{#1}}
\newcommand{\VerbatimStringTok}[1]{\textcolor[rgb]{0.31,0.60,0.02}{#1}}
\newcommand{\WarningTok}[1]{\textcolor[rgb]{0.56,0.35,0.01}{\textbf{\textit{#1}}}}
\usepackage{graphicx}
\makeatletter
\def\maxwidth{\ifdim\Gin@nat@width>\linewidth\linewidth\else\Gin@nat@width\fi}
\def\maxheight{\ifdim\Gin@nat@height>\textheight\textheight\else\Gin@nat@height\fi}
\makeatother
% Scale images if necessary, so that they will not overflow the page
% margins by default, and it is still possible to overwrite the defaults
% using explicit options in \includegraphics[width, height, ...]{}
\setkeys{Gin}{width=\maxwidth,height=\maxheight,keepaspectratio}
% Set default figure placement to htbp
\makeatletter
\def\fps@figure{htbp}
\makeatother
\setlength{\emergencystretch}{3em} % prevent overfull lines
\providecommand{\tightlist}{%
  \setlength{\itemsep}{0pt}\setlength{\parskip}{0pt}}
\setcounter{secnumdepth}{-\maxdimen} % remove section numbering
\ifluatex
  \usepackage{selnolig}  % disable illegal ligatures
\fi

\title{젊은 임상 연구자들을 위한 필수 R}
\author{Jonggi Choi}
\date{March 14, 2021}

\begin{document}
\maketitle

본 문서는 배포는 가능하지만 출처는 밝혀주시길 바랍니다.

\textbf{작성자}\\
울산의대 서울아산병원\\
소화기내과 조교수 최종기\\
작성 일시: March 14, 2021\\
버전: Version 1.0

\hypertarget{basic-function-for-data-exploration}{%
\subsection{1. Basic function for data
exploration}\label{basic-function-for-data-exploration}}

\hypertarget{uxc2e4uxc2b5uxc5d0-uxd544uxc694uxd55c-package-loading}{%
\subsubsection{1.1 실습에 필요한 package
loading}\label{uxc2e4uxc2b5uxc5d0-uxd544uxc694uxd55c-package-loading}}

\begin{Shaded}
\begin{Highlighting}[]
\FunctionTok{library}\NormalTok{(tidyverse)}
\end{Highlighting}
\end{Shaded}

\hypertarget{data-uxbd88uxb7ecuxc624uxae30}{%
\subsubsection{1.2 Data 불러오기}\label{data-uxbd88uxb7ecuxc624uxae30}}

\begin{itemize}
\tightlist
\item
  \textbf{현재 사용하고 있는 working directory 확인}
\end{itemize}

\begin{Shaded}
\begin{Highlighting}[]
\FunctionTok{getwd}\NormalTok{()}
\end{Highlighting}
\end{Shaded}

\begin{verbatim}
[1] "/Users/ChoiJongGi/Dropbox/Data science study/R_JonggiChoi"
\end{verbatim}

Mac과 Windows는 폴더 경로 표시 방식이 다르기때문에 주의!

\begin{itemize}
\tightlist
\item
  \textbf{실습을 위해 사용할 data 불러오기 (chb\_example.csv)}
\end{itemize}

\begin{Shaded}
\begin{Highlighting}[]
\CommentTok{\# Data (csv파일)를 불러오는 방식에는 2가지가 있다.}
\CommentTok{\# 일반 data.frame으로 불러오는 경우}
\NormalTok{dat }\OtherTok{\textless{}{-}} \FunctionTok{read.csv}\NormalTok{(}\StringTok{"chb\_example.csv"}\NormalTok{) }\CommentTok{\# read.csv 이용}
\FunctionTok{class}\NormalTok{(dat) }\CommentTok{\# data 형태가 data.frame}
\end{Highlighting}
\end{Shaded}

\begin{verbatim}
[1] "data.frame"
\end{verbatim}

\begin{Shaded}
\begin{Highlighting}[]
\CommentTok{\# Tibble형태로 불러오는 경우}
\NormalTok{dat }\OtherTok{\textless{}{-}} \FunctionTok{read\_csv}\NormalTok{(}\StringTok{"chb\_example.csv"}\NormalTok{) }\CommentTok{\# read\_csv 이용}
\FunctionTok{class}\NormalTok{(dat) }\CommentTok{\# data 형태가 tibble (tbl\_df)}
\end{Highlighting}
\end{Shaded}

\begin{verbatim}
[1] "spec_tbl_df" "tbl_df"      "tbl"         "data.frame" 
\end{verbatim}

일반적인 data.frame형태로 불러와도 되지만 tibble형태가 조금 더 깔끔하고
tidyverse 명령어에 적합하다.\\
하지만 기본적으로 tibble은 data.frame과 같은 속성에서 비롯되었다.

\begin{itemize}
\tightlist
\item
  \textbf{Data는 어떻게 생겼는지 구조 파악}
\end{itemize}

\begin{Shaded}
\begin{Highlighting}[]
\CommentTok{\# Data의 구조를 보여주는 기본 명령어}
\FunctionTok{str}\NormalTok{(dat) }
\end{Highlighting}
\end{Shaded}

\begin{verbatim}
spec_tbl_df [1,000 x 23] (S3: spec_tbl_df/tbl_df/tbl/data.frame)
 $ index_date : Date[1:1000], format: "2007-01-05" "2007-01-10" ...
 $ gender     : chr [1:1000] "M" "F" "M" "M" ...
 $ age        : num [1:1000] 54 45 49 26 50 33 49 50 49 50 ...
 $ last_date  : Date[1:1000], format: "2014-07-18" "2016-08-25" ...
 $ treat_gr   : chr [1:1000] "ETV" "ETV" "ETV" "ETV" ...
 $ lc         : num [1:1000] 1 0 1 0 1 1 1 0 1 0 ...
 $ dietpl     : num [1:1000] 0 0 0 0 0 0 0 0 0 0 ...
 $ dietpl_date: Date[1:1000], format: "2017-09-13" "2016-08-25" ...
 $ dietpl_yr  : num [1:1000] 10.84 9.76 10.59 6.02 2.36 ...
 $ hcc        : num [1:1000] 1 0 0 0 0 0 1 0 0 0 ...
 $ hcc_date   : Date[1:1000], format: "2014-07-18" "2016-08-25" ...
 $ hcc_yr     : num [1:1000] 7.64 9.76 10.59 6.02 2.36 ...
 $ b_lab_date : Date[1:1000], format: "2006-12-28" "2006-12-28" ...
 $ b_alt      : num [1:1000] 67 32 106 159 94 32 104 143 31 239 ...
 $ b_bil      : num [1:1000] 1.2 1.1 1.8 1.4 0.8 0.7 1.1 1.5 1.3 1.3 ...
 $ b_inr      : num [1:1000] 1.17 0.9 1.03 1.04 1.12 0.88 1.1 NA 1.04 1.03 ...
 $ b_cr       : num [1:1000] 1 0.6 1 0.93 0.9 1.2 0.9 0.7 0.7 1.2 ...
 $ b_plt      : num [1:1000] 110 187 133 164 153 170 NA 165 157 166 ...
 $ b_alb      : num [1:1000] 3.8 3.8 3.9 4.2 4.2 3.4 3.7 3.5 3.9 3.2 ...
 $ b_eag      : num [1:1000] 1 0 0 0 1 1 1 1 1 1 ...
 $ b_eab      : num [1:1000] 0 1 1 1 0 0 0 NA 0 0 ...
 $ b_dna      : chr [1:1000] "31934936.8" "1100" "35968299.4" "9145608.2" ...
 $ b_dna_log  : num [1:1000] 7.5 3.04 7.56 6.96 6.04 ...
 - attr(*, "spec")=
  .. cols(
  ..   index_date = col_date(format = ""),
  ..   gender = col_character(),
  ..   age = col_double(),
  ..   last_date = col_date(format = ""),
  ..   treat_gr = col_character(),
  ..   lc = col_double(),
  ..   dietpl = col_double(),
  ..   dietpl_date = col_date(format = ""),
  ..   dietpl_yr = col_double(),
  ..   hcc = col_double(),
  ..   hcc_date = col_date(format = ""),
  ..   hcc_yr = col_double(),
  ..   b_lab_date = col_date(format = ""),
  ..   b_alt = col_double(),
  ..   b_bil = col_double(),
  ..   b_inr = col_double(),
  ..   b_cr = col_double(),
  ..   b_plt = col_double(),
  ..   b_alb = col_double(),
  ..   b_eag = col_double(),
  ..   b_eab = col_double(),
  ..   b_dna = col_character(),
  ..   b_dna_log = col_double()
  .. )
\end{verbatim}

\begin{Shaded}
\begin{Highlighting}[]
\CommentTok{\# 제일 위 6줄만 보여줌. 많이 쓰임}
\FunctionTok{head}\NormalTok{(dat)}
\end{Highlighting}
\end{Shaded}

\begin{verbatim}
# A tibble: 6 x 23
  index_date gender   age last_date  treat_gr    lc dietpl dietpl_date dietpl_yr
  <date>     <chr>  <dbl> <date>     <chr>    <dbl>  <dbl> <date>          <dbl>
1 2007-01-05 M         54 2014-07-18 ETV          1      0 2017-09-13      10.8 
2 2007-01-10 F         45 2016-08-25 ETV          0      0 2016-08-25       9.76
3 2007-01-11 M         49 2017-06-21 ETV          1      0 2017-06-21      10.6 
4 2007-01-12 M         26 2012-12-17 ETV          0      0 2012-12-17       6.02
5 2007-01-18 M         50 2009-05-15 ETV          1      0 2009-05-15       2.36
6 2007-01-18 M         33 2013-01-11 ETV          1      0 2013-01-11       6.07
# ... with 14 more variables: hcc <dbl>, hcc_date <date>, hcc_yr <dbl>,
#   b_lab_date <date>, b_alt <dbl>, b_bil <dbl>, b_inr <dbl>, b_cr <dbl>,
#   b_plt <dbl>, b_alb <dbl>, b_eag <dbl>, b_eab <dbl>, b_dna <chr>,
#   b_dna_log <dbl>
\end{verbatim}

\begin{Shaded}
\begin{Highlighting}[]
\CommentTok{\# 이것도 가능, 3줄만 보여줌}
\FunctionTok{head}\NormalTok{(dat, }\DecValTok{3}\NormalTok{)}
\end{Highlighting}
\end{Shaded}

\begin{verbatim}
# A tibble: 3 x 23
  index_date gender   age last_date  treat_gr    lc dietpl dietpl_date dietpl_yr
  <date>     <chr>  <dbl> <date>     <chr>    <dbl>  <dbl> <date>          <dbl>
1 2007-01-05 M         54 2014-07-18 ETV          1      0 2017-09-13      10.8 
2 2007-01-10 F         45 2016-08-25 ETV          0      0 2016-08-25       9.76
3 2007-01-11 M         49 2017-06-21 ETV          1      0 2017-06-21      10.6 
# ... with 14 more variables: hcc <dbl>, hcc_date <date>, hcc_yr <dbl>,
#   b_lab_date <date>, b_alt <dbl>, b_bil <dbl>, b_inr <dbl>, b_cr <dbl>,
#   b_plt <dbl>, b_alb <dbl>, b_eag <dbl>, b_eab <dbl>, b_dna <chr>,
#   b_dna_log <dbl>
\end{verbatim}

\begin{Shaded}
\begin{Highlighting}[]
\CommentTok{\# 제일 마지막 6줄 보여줌}
\FunctionTok{tail}\NormalTok{(dat)}
\end{Highlighting}
\end{Shaded}

\begin{verbatim}
# A tibble: 6 x 23
  index_date gender   age last_date  treat_gr    lc dietpl dietpl_date dietpl_yr
  <date>     <chr>  <dbl> <date>     <chr>    <dbl>  <dbl> <date>          <dbl>
1 2011-01-04 M         58 2017-10-19 ETV          1      0 2017-10-19       6.89
2 2011-01-05 F         49 2017-10-31 ETV          1      0 2017-10-31       6.92
3 2011-01-07 M         44 2017-05-08 ETV          1      0 2017-05-08       6.42
4 2011-01-07 M         48 2017-04-06 ETV          0      0 2017-04-06       6.34
5 2011-01-12 M         64 2013-03-31 ETV          1      0 2017-10-24       6.88
6 2011-01-12 F         51 2017-10-31 ETV          1      0 2017-10-31       6.9 
# ... with 14 more variables: hcc <dbl>, hcc_date <date>, hcc_yr <dbl>,
#   b_lab_date <date>, b_alt <dbl>, b_bil <dbl>, b_inr <dbl>, b_cr <dbl>,
#   b_plt <dbl>, b_alb <dbl>, b_eag <dbl>, b_eab <dbl>, b_dna <chr>,
#   b_dna_log <dbl>
\end{verbatim}

\begin{Shaded}
\begin{Highlighting}[]
\CommentTok{\# Tibble 형태의 data는 glimpse 형태 명령어를 쓰면 깔끔하게 요약되어 보인다}
\FunctionTok{glimpse}\NormalTok{(dat)}
\end{Highlighting}
\end{Shaded}

\begin{verbatim}
Rows: 1,000
Columns: 23
$ index_date  <date> 2007-01-05, 2007-01-10, 2007-01-11, 2007-01-12, 2007-0...
$ gender      <chr> "M", "F", "M", "M", "M", "M", "M", "F", "F", "M", "M", ...
$ age         <dbl> 54, 45, 49, 26, 50, 33, 49, 50, 49, 50, 32, 48, 51, 41,...
$ last_date   <date> 2014-07-18, 2016-08-25, 2017-06-21, 2012-12-17, 2009-0...
$ treat_gr    <chr> "ETV", "ETV", "ETV", "ETV", "ETV", "ETV", "ETV", "ETV",...
$ lc          <dbl> 1, 0, 1, 0, 1, 1, 1, 0, 1, 0, 0, 0, 0, 1, 1, 1, 1, 1, 0...
$ dietpl      <dbl> 0, 0, 0, 0, 0, 0, 0, 0, 0, 0, 0, 0, 0, 0, 0, 0, 0, 0, 0...
$ dietpl_date <date> 2017-09-13, 2016-08-25, 2017-06-21, 2012-12-17, 2009-0...
$ dietpl_yr   <dbl> 10.844444, 9.763889, 10.594444, 6.016667, 2.355556, 6.0...
$ hcc         <dbl> 1, 0, 0, 0, 0, 0, 1, 0, 0, 0, 0, 1, 0, 0, 1, 0, 0, 0, 0...
$ hcc_date    <date> 2014-07-18, 2016-08-25, 2017-06-21, 2012-12-17, 2009-0...
$ hcc_yr      <dbl> 7.641667, 9.763889, 10.594444, 6.016667, 2.355556, 6.06...
$ b_lab_date  <date> 2006-12-28, 2006-12-28, 2006-12-28, 2007-01-06, 2007-0...
$ b_alt       <dbl> 67, 32, 106, 159, 94, 32, 104, 143, 31, 239, 359, 486, ...
$ b_bil       <dbl> 1.2, 1.1, 1.8, 1.4, 0.8, 0.7, 1.1, 1.5, 1.3, 1.3, 0.9, ...
$ b_inr       <dbl> 1.17, 0.90, 1.03, 1.04, 1.12, 0.88, 1.10, NA, 1.04, 1.0...
$ b_cr        <dbl> 1.00, 0.60, 1.00, 0.93, 0.90, 1.20, 0.90, 0.70, 0.70, 1...
$ b_plt       <dbl> 110, 187, 133, 164, 153, 170, NA, 165, 157, 166, 202, 1...
$ b_alb       <dbl> 3.8, 3.8, 3.9, 4.2, 4.2, 3.4, 3.7, 3.5, 3.9, 3.2, 4.1, ...
$ b_eag       <dbl> 1, 0, 0, 0, 1, 1, 1, 1, 1, 1, 0, 0, 1, 0, 1, 1, 1, 0, 0...
$ b_eab       <dbl> 0, 1, 1, 1, 0, 0, 0, NA, 0, 0, 1, 1, 0, 1, 0, 0, 0, 1, ...
$ b_dna       <chr> "31934936.8", "1100", "35968299.4", "9145608.2", "10954...
$ b_dna_log   <dbl> 7.504266, 3.041393, 7.555920, 6.961213, 6.039605, 2.954...
\end{verbatim}

\hypertarget{data-uxd2b9uxc9d5-uxd30cuxc545}{%
\subsubsection{1.3 Data 특징
파악}\label{data-uxd2b9uxc9d5-uxd30cuxc545}}

\begin{itemize}
\tightlist
\item
  \textbf{Data의 구조 및 크기 파악}
\end{itemize}

\begin{Shaded}
\begin{Highlighting}[]
\CommentTok{\# 전체 data dimension}
\FunctionTok{dim}\NormalTok{(dat)}
\end{Highlighting}
\end{Shaded}

\begin{verbatim}
[1] 1000   23
\end{verbatim}

1000행(케이스 혹은 환자), 23열(컬럼 혹은 variable)

\begin{Shaded}
\begin{Highlighting}[]
\CommentTok{\# 전체 케이스 숫자만 궁금할때}
\FunctionTok{nrow}\NormalTok{(dat)}
\end{Highlighting}
\end{Shaded}

\begin{verbatim}
[1] 1000
\end{verbatim}

\begin{Shaded}
\begin{Highlighting}[]
\CommentTok{\# column name (=variable name)}
\FunctionTok{colnames}\NormalTok{(dat)}
\end{Highlighting}
\end{Shaded}

\begin{verbatim}
 [1] "index_date"  "gender"      "age"         "last_date"   "treat_gr"   
 [6] "lc"          "dietpl"      "dietpl_date" "dietpl_yr"   "hcc"        
[11] "hcc_date"    "hcc_yr"      "b_lab_date"  "b_alt"       "b_bil"      
[16] "b_inr"       "b_cr"        "b_plt"       "b_alb"       "b_eag"      
[21] "b_eab"       "b_dna"       "b_dna_log"  
\end{verbatim}

\begin{itemize}
\tightlist
\item
  Data variable summary

  \begin{itemize}
  \tightlist
  \item
    index\_date: 투약 시작 시점
  \item
    gender: 성별
  \item
    age: 나이
  \item
    last\_date: 마지막 추적관찰
  \item
    treat\_gr: 항바이러스제 종류 (ETV: entecavir, TDF: tenofovir)
  \item
    lc: liver cirrhosis
  \item
    dietpl: Death or liver transplantation
  \item
    dietpl\_date: Date of death or liver transplantation
  \item
    dietpl\_yr: Time interval from index date to death/transplantation
  \item
    hcc: HCC
  \item
    hcc\_date: Date of HCC diagnosis
  \item
    hcc\_yr: Time interval from index date to HCC diagnosis
  \item
    b\_lab\_date: Baseline lab 시행 일자
  \item
    b\_alt: Baseline ALT
  \item
    b\_bil: Baseline total bilirubin
  \item
    b\_inr: Baseline PT(INR)
  \item
    b\_cr: Baseline creatinine
  \item
    b\_plt: Baseline platelet count
  \item
    b\_alb: Baseline albumin
  \item
    b\_eag: Baseline HBeAg status
  \item
    b\_eab: Baseline HBeAb status
  \item
    b\_dna: Baseline HBV DNA level
  \item
    b\_dna\_log: Baseline HBV DNA level (log)\\
  \end{itemize}
\item
  \textbf{각 variable들의 특징 파악}
\end{itemize}

\begin{Shaded}
\begin{Highlighting}[]
\CommentTok{\# Age 변수 특성 파악}
\CommentTok{\# 평균}
\FunctionTok{mean}\NormalTok{(dat}\SpecialCharTok{$}\NormalTok{age)}
\end{Highlighting}
\end{Shaded}

\begin{verbatim}
[1] 46.062
\end{verbatim}

\begin{Shaded}
\begin{Highlighting}[]
\CommentTok{\# 중앙값}
\FunctionTok{median}\NormalTok{(dat}\SpecialCharTok{$}\NormalTok{age)}
\end{Highlighting}
\end{Shaded}

\begin{verbatim}
[1] 47
\end{verbatim}

\begin{Shaded}
\begin{Highlighting}[]
\CommentTok{\# 표준편차}
\FunctionTok{sd}\NormalTok{(dat}\SpecialCharTok{$}\NormalTok{age)}
\end{Highlighting}
\end{Shaded}

\begin{verbatim}
[1] 10.20825
\end{verbatim}

\begin{Shaded}
\begin{Highlighting}[]
\CommentTok{\# 분산}
\FunctionTok{var}\NormalTok{(dat}\SpecialCharTok{$}\NormalTok{age)}
\end{Highlighting}
\end{Shaded}

\begin{verbatim}
[1] 104.2084
\end{verbatim}

\begin{Shaded}
\begin{Highlighting}[]
\CommentTok{\# 최소값}
\FunctionTok{min}\NormalTok{(dat}\SpecialCharTok{$}\NormalTok{age)}
\end{Highlighting}
\end{Shaded}

\begin{verbatim}
[1] 19
\end{verbatim}

\begin{Shaded}
\begin{Highlighting}[]
\CommentTok{\# 최대값}
\FunctionTok{max}\NormalTok{(dat}\SpecialCharTok{$}\NormalTok{age)}
\end{Highlighting}
\end{Shaded}

\begin{verbatim}
[1] 74
\end{verbatim}

\begin{Shaded}
\begin{Highlighting}[]
\CommentTok{\# 범위}
\FunctionTok{range}\NormalTok{(dat}\SpecialCharTok{$}\NormalTok{age)}
\end{Highlighting}
\end{Shaded}

\begin{verbatim}
[1] 19 74
\end{verbatim}

\begin{Shaded}
\begin{Highlighting}[]
\CommentTok{\# 1사분위수{-}3사분위수 범위}
\FunctionTok{IQR}\NormalTok{(dat}\SpecialCharTok{$}\NormalTok{age)}
\end{Highlighting}
\end{Shaded}

\begin{verbatim}
[1] 13
\end{verbatim}

\begin{Shaded}
\begin{Highlighting}[]
\CommentTok{\# 한꺼번에 보여주기}
\FunctionTok{summary}\NormalTok{(dat}\SpecialCharTok{$}\NormalTok{age)}
\end{Highlighting}
\end{Shaded}

\begin{verbatim}
   Min. 1st Qu.  Median    Mean 3rd Qu.    Max. 
  19.00   40.00   47.00   46.06   53.00   74.00 
\end{verbatim}

\begin{Shaded}
\begin{Highlighting}[]
\CommentTok{\# Hmisc package의 describe 명령어가 유용}
\CommentTok{\# Age 변수의 특징 파악}
\NormalTok{Hmisc}\SpecialCharTok{::}\FunctionTok{describe}\NormalTok{(dat}\SpecialCharTok{$}\NormalTok{age)}
\end{Highlighting}
\end{Shaded}

\begin{verbatim}
dat$age 
       n  missing distinct     Info     Mean      Gmd      .05      .10 
    1000        0       55    0.999    46.06    11.54       27       32 
     .25      .50      .75      .90      .95 
      40       47       53       59       62 

lowest : 19 20 21 22 23, highest: 70 71 72 73 74
\end{verbatim}

\begin{itemize}
\tightlist
\item
  \textbf{결측값이 존재하는 경우 데이터 특성 파악}
\end{itemize}

\begin{Shaded}
\begin{Highlighting}[]
\CommentTok{\# 결측값이 존재하는 variable은 평균 계산 불가}
\FunctionTok{mean}\NormalTok{(dat}\SpecialCharTok{$}\NormalTok{b\_alb) }\CommentTok{\# NA로 표시됨}
\end{Highlighting}
\end{Shaded}

\begin{verbatim}
[1] NA
\end{verbatim}

\begin{Shaded}
\begin{Highlighting}[]
\CommentTok{\# 결측값이 존재하는 경우 \textquotesingle{}na.rm=TRUE\textquotesingle{} option 추가}
\FunctionTok{mean}\NormalTok{(dat}\SpecialCharTok{$}\NormalTok{b\_alb, }\AttributeTok{na.rm=}\NormalTok{T) }\CommentTok{\# na.rm의미 = NA remove}
\end{Highlighting}
\end{Shaded}

\begin{verbatim}
[1] 3.734274
\end{verbatim}

NA를 제외하고 mean을 계산을 해준다

\begin{itemize}
\tightlist
\item
  \textbf{결측값(NA)이 존재하는지 확인하기}
\end{itemize}

\begin{Shaded}
\begin{Highlighting}[]
\CommentTok{\# NA 존재 명령어 \textquotesingle{}is.na\textquotesingle{}}
\CommentTok{\# is.na 결과값이 True = 결측값}
\CommentTok{\# is.na 결과값이 False = 결측값 아님}
\FunctionTok{is.na}\NormalTok{(dat}\SpecialCharTok{$}\NormalTok{b\_inr[}\DecValTok{1}\SpecialCharTok{:}\DecValTok{20}\NormalTok{])}
\end{Highlighting}
\end{Shaded}

\begin{verbatim}
 [1] FALSE FALSE FALSE FALSE FALSE FALSE FALSE  TRUE FALSE FALSE FALSE FALSE
[13] FALSE FALSE FALSE FALSE FALSE FALSE FALSE FALSE
\end{verbatim}

'b\_inr'변수의 1-20행까지 결측값이 존재하는지 확인가능

\begin{Shaded}
\begin{Highlighting}[]
\CommentTok{\# 전체 1,000개 행중에 b\_inr 변수의 총 NA는 몇개?}
\FunctionTok{sum}\NormalTok{(}\FunctionTok{is.na}\NormalTok{(dat}\SpecialCharTok{$}\NormalTok{b\_inr))}
\end{Highlighting}
\end{Shaded}

\begin{verbatim}
[1] 16
\end{verbatim}

b\_inr이 NA일경우 true (1로 게산됨) 이경우를 모두 sum하게 되니깐 총
16개의 NA가 b\_inr에 존재함

\begin{Shaded}
\begin{Highlighting}[]
\CommentTok{\# summary로도 확인가능}
\FunctionTok{summary}\NormalTok{(dat}\SpecialCharTok{$}\NormalTok{b\_inr)}
\end{Highlighting}
\end{Shaded}

\begin{verbatim}
   Min. 1st Qu.  Median    Mean 3rd Qu.    Max.    NA's 
  0.840   1.030   1.090   1.134   1.180   3.130      16 
\end{verbatim}

\begin{itemize}
\tightlist
\item
  \textbf{Data 형태 (class) 확인하기}
\end{itemize}

\begin{Shaded}
\begin{Highlighting}[]
\CommentTok{\# is.*() 함수를 이용한다.}

\FunctionTok{is.data.frame}\NormalTok{(dat)  }\CommentTok{\# data.frame?}
\end{Highlighting}
\end{Shaded}

\begin{verbatim}
[1] TRUE
\end{verbatim}

\begin{Shaded}
\begin{Highlighting}[]
\FunctionTok{is.matrix}\NormalTok{(dat)  }\CommentTok{\# matrix?}
\end{Highlighting}
\end{Shaded}

\begin{verbatim}
[1] FALSE
\end{verbatim}

\begin{Shaded}
\begin{Highlighting}[]
\FunctionTok{is.character}\NormalTok{(dat}\SpecialCharTok{$}\NormalTok{age)  }\CommentTok{\# 문자형 데이터인 범주형 변수? }
\end{Highlighting}
\end{Shaded}

\begin{verbatim}
[1] FALSE
\end{verbatim}

\begin{Shaded}
\begin{Highlighting}[]
\FunctionTok{is.character}\NormalTok{(dat}\SpecialCharTok{$}\NormalTok{gender)  }\CommentTok{\# 문자형 데이터인 범주형 변수? }
\end{Highlighting}
\end{Shaded}

\begin{verbatim}
[1] TRUE
\end{verbatim}

\begin{Shaded}
\begin{Highlighting}[]
\FunctionTok{is.numeric}\NormalTok{(dat}\SpecialCharTok{$}\NormalTok{age)  }\CommentTok{\# 수치형 데이터인 연속형 변수?}
\end{Highlighting}
\end{Shaded}

\begin{verbatim}
[1] TRUE
\end{verbatim}

\begin{Shaded}
\begin{Highlighting}[]
\FunctionTok{is.numeric}\NormalTok{(dat}\SpecialCharTok{$}\NormalTok{gender)  }\CommentTok{\# 수치형 데이터인 연속형 변수?}
\end{Highlighting}
\end{Shaded}

\begin{verbatim}
[1] FALSE
\end{verbatim}

\begin{Shaded}
\begin{Highlighting}[]
\FunctionTok{is.integer}\NormalTok{(dat}\SpecialCharTok{$}\NormalTok{age)  }\CommentTok{\# 정수형 데이터인 연속형 변수?}
\end{Highlighting}
\end{Shaded}

\begin{verbatim}
[1] FALSE
\end{verbatim}

\begin{Shaded}
\begin{Highlighting}[]
\FunctionTok{is.double}\NormalTok{(dat}\SpecialCharTok{$}\NormalTok{age)  }\CommentTok{\# 더블형 데이터인 연속형 변수? }
\end{Highlighting}
\end{Shaded}

\begin{verbatim}
[1] TRUE
\end{verbatim}

\textless br?\textgreater{}

\hypertarget{baseline-characteristics-table-uxb9ccuxb4e4uxae30}{%
\subsubsection{1.4 Baseline characteristics Table
만들기}\label{baseline-characteristics-table-uxb9ccuxb4e4uxae30}}

\begin{Shaded}
\begin{Highlighting}[]
\CommentTok{\# moonBook package를 이용하여 descriptive analysis}
\CommentTok{\# treat\_gr (ETV:entecavir, TDF:tenofovir)에 따른 두군의 \# 비교}
\FunctionTok{library}\NormalTok{(moonBook)}
\FunctionTok{mytable}\NormalTok{(treat\_gr}\SpecialCharTok{\textasciitilde{}}\NormalTok{age}\SpecialCharTok{+}\NormalTok{gender,}\AttributeTok{data=}\NormalTok{dat)}
\end{Highlighting}
\end{Shaded}

\begin{verbatim}
Descriptive Statistics by 'treat_gr'
_____________________________________ 
            ETV         TDF       p  
          (N=506)     (N=494)  
------------------------------------- 
 age    46.0 ± 10.2 46.1 ± 10.2 0.822
 gender                         0.178
   - F  150 (29.6%) 167 (33.8%)      
   - M  356 (70.4%) 327 (66.2%)      
------------------------------------- 
\end{verbatim}

\begin{Shaded}
\begin{Highlighting}[]
\CommentTok{\# 전체 그룹도 표시하기}
\FunctionTok{mytable}\NormalTok{(treat\_gr}\SpecialCharTok{\textasciitilde{}}\NormalTok{age}\SpecialCharTok{+}\NormalTok{gender, }\AttributeTok{data=}\NormalTok{dat, }\AttributeTok{show.total=}\ConstantTok{TRUE}\NormalTok{)}
\end{Highlighting}
\end{Shaded}

\begin{verbatim}
      Descriptive Statistics by 'treat_gr'      
_________________________________________________ 
            ETV         TDF        Total      p  
          (N=506)     (N=494)    (N=1000)  
------------------------------------------------- 
 age    46.0 ± 10.2 46.1 ± 10.2 46.1 ± 10.2 0.822
 gender                                     0.178
   - F  150 (29.6%) 167 (33.8%) 317 (31.7%)      
   - M  356 (70.4%) 327 (66.2%) 683 (68.3%)      
------------------------------------------------- 
\end{verbatim}

\begin{Shaded}
\begin{Highlighting}[]
\CommentTok{\# mean +/{-} SD 말고 median [IQR]로 표시하기}
\FunctionTok{mytable}\NormalTok{(treat\_gr}\SpecialCharTok{\textasciitilde{}}\NormalTok{age}\SpecialCharTok{+}\NormalTok{gender,}\AttributeTok{data=}\NormalTok{dat, }\AttributeTok{method=}\DecValTok{2}\NormalTok{)}
\end{Highlighting}
\end{Shaded}

\begin{verbatim}
     Descriptive Statistics by 'treat_gr'     
_______________________________________________ 
              ETV              TDF          p  
            (N=506)          (N=494)     
----------------------------------------------- 
 age    47.0 [40.0;53.0] 46.0 [40.0;53.0] 0.979
 gender                                   0.178
   - F    150 (29.6%)      167 (33.8%)         
   - M    356 (70.4%)      327 (66.2%)         
----------------------------------------------- 
\end{verbatim}

\begin{Shaded}
\begin{Highlighting}[]
\CommentTok{\# treat\_gr에 따른 모든 변수 비교}
\FunctionTok{mytable}\NormalTok{(treat\_gr}\SpecialCharTok{\textasciitilde{}}\NormalTok{., }\AttributeTok{data=}\NormalTok{dat) }\CommentTok{\# \textasciitilde{}. 의 의미는 \textquotesingle{}treat\_gr\textquotesingle{}제외하고 나머지 모든 변수를 넣어달라는 의미}
\end{Highlighting}
\end{Shaded}

\begin{verbatim}
             Descriptive Statistics by 'treat_gr'            
______________________________________________________________ 
                      ETV                   TDF            p  
                    (N=506)               (N=494)       
-------------------------------------------------------------- 
 index_date  2007-01-05-2011-01-12 2007-01-05-2011-01-12      
 gender                                                  0.178
   - F            150 (29.6%)           167 (33.8%)           
   - M            356 (70.4%)           327 (66.2%)           
 age              46.0 ± 10.2           46.1 ± 10.2      0.822
 last_date   2007-12-18-2017-11-01 2007-12-18-2017-11-01      
 lc                                                      0.498
   - 0            234 (46.2%)           240 (48.6%)           
   - 1            272 (53.8%)           254 (51.4%)           
 dietpl                                                  0.072
   - 0            492 (97.2%)           489 (99.0%)           
   - 1            14 ( 2.8%)             5 ( 1.0%)            
 dietpl_date 2009-01-28-2017-11-01 2009-01-28-2017-11-01      
 dietpl_yr         7.4 ±  2.8            6.7 ±  2.2      0.000
 hcc                                                     0.050
   - 0            418 (82.6%)           431 (87.2%)           
   - 1            88 (17.4%)            63 (12.8%)            
 hcc_date    2007-12-18-2017-11-01 2007-12-18-2017-11-01      
 hcc_yr            6.8 ±  3.1            6.2 ±  2.4      0.003
 b_lab_date  2006-12-28-2011-01-17 2006-12-28-2011-01-17      
 b_alt           195.9 ± 257.7         203.2 ± 303.5     0.682
 b_bil             1.7 ±  2.1            1.9 ±  3.4      0.158
 b_inr             1.1 ±  0.2            1.2 ±  0.2      0.003
 b_cr              0.9 ±  0.5            1.0 ±  0.8      0.152
 b_plt           148.5 ± 58.8          149.0 ± 58.5      0.894
 b_alb             3.7 ±  0.5            3.8 ±  0.6      0.239
 b_eag                                                   0.009
   - 0            159 (31.4%)           195 (39.5%)           
   - 1            347 (68.6%)           299 (60.5%)           
 b_eab                                                   0.024
   - 0            300 (68.6%)           257 (61.0%)           
   - 1            137 (31.4%)           164 (39.0%)           
 b_dna         unique values:466     unique values:466        
 b_dna_log         6.9 ±  1.9            6.4 ±  2.0      0.000
-------------------------------------------------------------- 
\end{verbatim}

\begin{Shaded}
\begin{Highlighting}[]
\CommentTok{\# 만든 table 저장하기}
\NormalTok{table1}\OtherTok{\textless{}{-}}\FunctionTok{mytable}\NormalTok{(treat\_gr}\SpecialCharTok{\textasciitilde{}}\NormalTok{age}\SpecialCharTok{+}\NormalTok{gender, }\AttributeTok{data=}\NormalTok{dat)}
\NormalTok{table1}
\end{Highlighting}
\end{Shaded}

\begin{verbatim}
Descriptive Statistics by 'treat_gr'
_____________________________________ 
            ETV         TDF       p  
          (N=506)     (N=494)  
------------------------------------- 
 age    46.0 ± 10.2 46.1 ± 10.2 0.822
 gender                         0.178
   - F  150 (29.6%) 167 (33.8%)      
   - M  356 (70.4%) 327 (66.2%)      
------------------------------------- 
\end{verbatim}

\begin{Shaded}
\begin{Highlighting}[]
\FunctionTok{mycsv}\NormalTok{(table1, }\AttributeTok{file=}\StringTok{"table1.csv"}\NormalTok{) }
\CommentTok{\# mycsv(table객체, file="파일이름")}
\end{Highlighting}
\end{Shaded}

특별히 경로를 지정해주지 않으면현재 working directory에 저장이 된다.

\hypertarget{data-manipulation-using-dplyr}{%
\subsection{2. Data manipulation using
dplyr}\label{data-manipulation-using-dplyr}}

\hypertarget{key-five-functions-in-tidyverse}{%
\subsubsection{2.1 Key five functions in
Tidyverse}\label{key-five-functions-in-tidyverse}}

\begin{enumerate}
\def\labelenumi{\arabic{enumi}.}
\tightlist
\item
  filter
\item
  select
\item
  arrange
\item
  mutate
\item
  summarise
\end{enumerate}

\hypertarget{filter-function}{%
\paragraph{2.1.1 Filter function}\label{filter-function}}

\begin{Shaded}
\begin{Highlighting}[]
\CommentTok{\# 원본 data는 손대지말고 연습을 위해 복사본 (dat1)을 만들자}
\NormalTok{dat1 }\OtherTok{\textless{}{-}}\NormalTok{ dat}
\end{Highlighting}
\end{Shaded}

\textbf{R은 되돌아가기 (Ctrl+Z)기능이 없기에, data의 변형이 우려되면
복사본을 만들어두고 가능하면 original data는 그대로 두는것이 좋다 }

\begin{itemize}
\tightlist
\item
  filter는 말그대로 필터다. 엑셀의 그 깔대기모양의 `필터'
\item
  즉 필요한 조건에 만족하는 값만을 출력
\end{itemize}

\begin{Shaded}
\begin{Highlighting}[]
\CommentTok{\# Gender (Male: M or Female: F)를 이용}
\NormalTok{dat1 }\SpecialCharTok{\%\textgreater{}\%} 
  \FunctionTok{count}\NormalTok{(gender)}
\end{Highlighting}
\end{Shaded}

\begin{verbatim}
# A tibble: 2 x 2
  gender     n
* <chr>  <int>
1 F        317
2 M        683
\end{verbatim}

\begin{itemize}
\tightlist
\item
  dat1에서 male값만을 선택
\end{itemize}

\begin{Shaded}
\begin{Highlighting}[]
\NormalTok{dat1 }\SpecialCharTok{\%\textgreater{}\%} 
   \FunctionTok{filter}\NormalTok{(gender }\SpecialCharTok{==} \StringTok{"M"}\NormalTok{) }\CommentTok{\# 같다는= 한개가 아니고 == 2개!}
\end{Highlighting}
\end{Shaded}

\begin{verbatim}
# A tibble: 683 x 23
   index_date gender   age last_date  treat_gr    lc dietpl dietpl_date
   <date>     <chr>  <dbl> <date>     <chr>    <dbl>  <dbl> <date>     
 1 2007-01-05 M         54 2014-07-18 ETV          1      0 2017-09-13 
 2 2007-01-11 M         49 2017-06-21 ETV          1      0 2017-06-21 
 3 2007-01-12 M         26 2012-12-17 ETV          0      0 2012-12-17 
 4 2007-01-18 M         50 2009-05-15 ETV          1      0 2009-05-15 
 5 2007-01-18 M         33 2013-01-11 ETV          1      0 2013-01-11 
 6 2007-01-26 M         49 2013-03-31 ETV          1      0 2017-08-07 
 7 2007-02-01 M         50 2017-09-12 ETV          0      0 2017-09-12 
 8 2007-02-01 M         32 2010-09-30 ETV          0      0 2010-09-30 
 9 2007-02-01 M         48 2010-04-23 ETV          0      0 2010-04-23 
10 2007-02-01 M         51 2017-08-03 ETV          0      0 2017-08-03 
# ... with 673 more rows, and 15 more variables: dietpl_yr <dbl>, hcc <dbl>,
#   hcc_date <date>, hcc_yr <dbl>, b_lab_date <date>, b_alt <dbl>, b_bil <dbl>,
#   b_inr <dbl>, b_cr <dbl>, b_plt <dbl>, b_alb <dbl>, b_eag <dbl>,
#   b_eab <dbl>, b_dna <chr>, b_dna_log <dbl>
\end{verbatim}

\begin{Shaded}
\begin{Highlighting}[]
\NormalTok{dat1 }\SpecialCharTok{\%\textgreater{}\%} 
   \FunctionTok{filter}\NormalTok{(gender }\SpecialCharTok{!=} \StringTok{"F"}\NormalTok{) }\CommentTok{\# 같지 않다는 != 로 표현}
\end{Highlighting}
\end{Shaded}

\begin{verbatim}
# A tibble: 683 x 23
   index_date gender   age last_date  treat_gr    lc dietpl dietpl_date
   <date>     <chr>  <dbl> <date>     <chr>    <dbl>  <dbl> <date>     
 1 2007-01-05 M         54 2014-07-18 ETV          1      0 2017-09-13 
 2 2007-01-11 M         49 2017-06-21 ETV          1      0 2017-06-21 
 3 2007-01-12 M         26 2012-12-17 ETV          0      0 2012-12-17 
 4 2007-01-18 M         50 2009-05-15 ETV          1      0 2009-05-15 
 5 2007-01-18 M         33 2013-01-11 ETV          1      0 2013-01-11 
 6 2007-01-26 M         49 2013-03-31 ETV          1      0 2017-08-07 
 7 2007-02-01 M         50 2017-09-12 ETV          0      0 2017-09-12 
 8 2007-02-01 M         32 2010-09-30 ETV          0      0 2010-09-30 
 9 2007-02-01 M         48 2010-04-23 ETV          0      0 2010-04-23 
10 2007-02-01 M         51 2017-08-03 ETV          0      0 2017-08-03 
# ... with 673 more rows, and 15 more variables: dietpl_yr <dbl>, hcc <dbl>,
#   hcc_date <date>, hcc_yr <dbl>, b_lab_date <date>, b_alt <dbl>, b_bil <dbl>,
#   b_inr <dbl>, b_cr <dbl>, b_plt <dbl>, b_alb <dbl>, b_eag <dbl>,
#   b_eab <dbl>, b_dna <chr>, b_dna_log <dbl>
\end{verbatim}

즉 위의 2개 코드는 같은 값을 출력 해준다.

\begin{itemize}
\tightlist
\item
  \textbf{복합 조건 (AND)}
\end{itemize}

\begin{Shaded}
\begin{Highlighting}[]
\CommentTok{\# older than 50 years old}
\NormalTok{dat1 }\SpecialCharTok{\%\textgreater{}\%} 
  \FunctionTok{filter}\NormalTok{(age}\SpecialCharTok{\textgreater{}=}\DecValTok{50}\NormalTok{) }\CommentTok{\# 385명}
\end{Highlighting}
\end{Shaded}

\begin{verbatim}
# A tibble: 385 x 23
   index_date gender   age last_date  treat_gr    lc dietpl dietpl_date
   <date>     <chr>  <dbl> <date>     <chr>    <dbl>  <dbl> <date>     
 1 2007-01-05 M         54 2014-07-18 ETV          1      0 2017-09-13 
 2 2007-01-18 M         50 2009-05-15 ETV          1      0 2009-05-15 
 3 2007-01-31 F         50 2017-08-24 ETV          0      0 2017-08-24 
 4 2007-02-01 M         50 2017-09-12 ETV          0      0 2017-09-12 
 5 2007-02-01 M         51 2017-08-03 ETV          0      0 2017-08-03 
 6 2007-02-08 M         50 2012-11-21 ETV          1      0 2012-11-21 
 7 2007-02-08 M         51 2011-01-21 ETV          1      0 2011-01-21 
 8 2007-02-09 F         62 2010-07-28 ETV          1      0 2010-07-28 
 9 2007-02-14 F         51 2009-06-17 ETV          1      0 2009-06-17 
10 2007-02-15 F         50 2012-01-31 ETV          1      1 2012-01-31 
# ... with 375 more rows, and 15 more variables: dietpl_yr <dbl>, hcc <dbl>,
#   hcc_date <date>, hcc_yr <dbl>, b_lab_date <date>, b_alt <dbl>, b_bil <dbl>,
#   b_inr <dbl>, b_cr <dbl>, b_plt <dbl>, b_alb <dbl>, b_eag <dbl>,
#   b_eab <dbl>, b_dna <chr>, b_dna_log <dbl>
\end{verbatim}

\begin{Shaded}
\begin{Highlighting}[]
\CommentTok{\# older than 50 years old + Male}
\NormalTok{dat1 }\SpecialCharTok{\%\textgreater{}\%} 
  \FunctionTok{filter}\NormalTok{(gender }\SpecialCharTok{==} \StringTok{"M"}\NormalTok{, age }\SpecialCharTok{\textgreater{}=}\DecValTok{50}\NormalTok{) }\CommentTok{\# 220명}
\end{Highlighting}
\end{Shaded}

\begin{verbatim}
# A tibble: 220 x 23
   index_date gender   age last_date  treat_gr    lc dietpl dietpl_date
   <date>     <chr>  <dbl> <date>     <chr>    <dbl>  <dbl> <date>     
 1 2007-01-05 M         54 2014-07-18 ETV          1      0 2017-09-13 
 2 2007-01-18 M         50 2009-05-15 ETV          1      0 2009-05-15 
 3 2007-02-01 M         50 2017-09-12 ETV          0      0 2017-09-12 
 4 2007-02-01 M         51 2017-08-03 ETV          0      0 2017-08-03 
 5 2007-02-08 M         50 2012-11-21 ETV          1      0 2012-11-21 
 6 2007-02-08 M         51 2011-01-21 ETV          1      0 2011-01-21 
 7 2007-02-15 M         52 2011-10-13 ETV          1      0 2011-10-13 
 8 2007-02-28 M         53 2008-12-25 ETV          1      1 2009-01-28 
 9 2007-03-09 M         51 2012-03-05 ETV          1      0 2012-03-05 
10 2007-03-15 M         52 2010-12-16 ETV          0      0 2010-12-16 
# ... with 210 more rows, and 15 more variables: dietpl_yr <dbl>, hcc <dbl>,
#   hcc_date <date>, hcc_yr <dbl>, b_lab_date <date>, b_alt <dbl>, b_bil <dbl>,
#   b_inr <dbl>, b_cr <dbl>, b_plt <dbl>, b_alb <dbl>, b_eag <dbl>,
#   b_eab <dbl>, b_dna <chr>, b_dna_log <dbl>
\end{verbatim}

\begin{Shaded}
\begin{Highlighting}[]
\CommentTok{\# older than 50 years old + Male + Cirrhosis}
\NormalTok{dat1 }\SpecialCharTok{\%\textgreater{}\%} 
  \FunctionTok{filter}\NormalTok{(gender }\SpecialCharTok{==} \StringTok{"M"}\NormalTok{, age }\SpecialCharTok{\textgreater{}=}\DecValTok{50}\NormalTok{, lc}\SpecialCharTok{==}\DecValTok{1}\NormalTok{) }\CommentTok{\# 152명}
\end{Highlighting}
\end{Shaded}

\begin{verbatim}
# A tibble: 152 x 23
   index_date gender   age last_date  treat_gr    lc dietpl dietpl_date
   <date>     <chr>  <dbl> <date>     <chr>    <dbl>  <dbl> <date>     
 1 2007-01-05 M         54 2014-07-18 ETV          1      0 2017-09-13 
 2 2007-01-18 M         50 2009-05-15 ETV          1      0 2009-05-15 
 3 2007-02-08 M         50 2012-11-21 ETV          1      0 2012-11-21 
 4 2007-02-08 M         51 2011-01-21 ETV          1      0 2011-01-21 
 5 2007-02-15 M         52 2011-10-13 ETV          1      0 2011-10-13 
 6 2007-02-28 M         53 2008-12-25 ETV          1      1 2009-01-28 
 7 2007-03-09 M         51 2012-03-05 ETV          1      0 2012-03-05 
 8 2007-03-22 M         54 2009-12-03 ETV          1      0 2009-12-03 
 9 2007-03-29 M         57 2014-08-22 ETV          1      0 2015-05-26 
10 2007-04-17 M         61 2017-08-21 ETV          1      0 2017-08-21 
# ... with 142 more rows, and 15 more variables: dietpl_yr <dbl>, hcc <dbl>,
#   hcc_date <date>, hcc_yr <dbl>, b_lab_date <date>, b_alt <dbl>, b_bil <dbl>,
#   b_inr <dbl>, b_cr <dbl>, b_plt <dbl>, b_alb <dbl>, b_eag <dbl>,
#   b_eab <dbl>, b_dna <chr>, b_dna_log <dbl>
\end{verbatim}

각 조건들간의 AND 조합은 , \& 어느것을 써도 된다.

\begin{itemize}
\tightlist
\item
  \textbf{복합 조건 (OR)}
\end{itemize}

\begin{Shaded}
\begin{Highlighting}[]
\CommentTok{\# age 30세 미만이거나 80세 이상인 경우}
\NormalTok{dat1 }\SpecialCharTok{\%\textgreater{}\%} 
  \FunctionTok{filter}\NormalTok{(age }\SpecialCharTok{\textless{}}\DecValTok{30} \SpecialCharTok{|}\NormalTok{ age }\SpecialCharTok{\textgreater{}}\DecValTok{80}\NormalTok{) }\CommentTok{\# 74명}
\end{Highlighting}
\end{Shaded}

\begin{verbatim}
# A tibble: 74 x 23
   index_date gender   age last_date  treat_gr    lc dietpl dietpl_date
   <date>     <chr>  <dbl> <date>     <chr>    <dbl>  <dbl> <date>     
 1 2007-01-12 M         26 2012-12-17 ETV          0      0 2012-12-17 
 2 2007-02-16 M         29 2017-07-14 ETV          0      0 2017-07-14 
 3 2007-03-02 M         25 2012-04-30 ETV          0      0 2012-04-30 
 4 2007-03-08 M         24 2017-04-12 ETV          0      0 2017-04-12 
 5 2007-04-02 F         22 2010-05-25 ETV          0      0 2010-05-25 
 6 2007-04-13 M         24 2015-09-22 ETV          0      0 2015-09-22 
 7 2007-04-16 M         29 2017-07-20 ETV          0      0 2017-07-20 
 8 2007-04-19 M         27 2009-02-27 ETV          0      0 2009-02-27 
 9 2007-05-03 M         26 2009-04-03 ETV          0      0 2009-04-03 
10 2007-07-03 F         24 2016-09-13 ETV          0      0 2016-09-13 
# ... with 64 more rows, and 15 more variables: dietpl_yr <dbl>, hcc <dbl>,
#   hcc_date <date>, hcc_yr <dbl>, b_lab_date <date>, b_alt <dbl>, b_bil <dbl>,
#   b_inr <dbl>, b_cr <dbl>, b_plt <dbl>, b_alb <dbl>, b_eag <dbl>,
#   b_eab <dbl>, b_dna <chr>, b_dna_log <dbl>
\end{verbatim}

\begin{Shaded}
\begin{Highlighting}[]
\CommentTok{\# age 70세 이상이거나 cirrhosis 있는 경우}
\NormalTok{dat1 }\SpecialCharTok{\%\textgreater{}\%} 
  \FunctionTok{filter}\NormalTok{(age }\SpecialCharTok{\textgreater{}}\DecValTok{70} \SpecialCharTok{|}\NormalTok{ lc}\SpecialCharTok{==}\DecValTok{1}\NormalTok{) }\CommentTok{\# 526명}
\end{Highlighting}
\end{Shaded}

\begin{verbatim}
# A tibble: 526 x 23
   index_date gender   age last_date  treat_gr    lc dietpl dietpl_date
   <date>     <chr>  <dbl> <date>     <chr>    <dbl>  <dbl> <date>     
 1 2007-01-05 M         54 2014-07-18 ETV          1      0 2017-09-13 
 2 2007-01-11 M         49 2017-06-21 ETV          1      0 2017-06-21 
 3 2007-01-18 M         50 2009-05-15 ETV          1      0 2009-05-15 
 4 2007-01-18 M         33 2013-01-11 ETV          1      0 2013-01-11 
 5 2007-01-26 M         49 2013-03-31 ETV          1      0 2017-08-07 
 6 2007-01-31 F         49 2009-03-20 ETV          1      0 2009-03-20 
 7 2007-02-02 M         41 2011-07-07 ETV          1      0 2011-07-07 
 8 2007-02-08 M         50 2012-11-21 ETV          1      0 2012-11-21 
 9 2007-02-08 M         51 2011-01-21 ETV          1      0 2011-01-21 
10 2007-02-09 F         62 2010-07-28 ETV          1      0 2010-07-28 
# ... with 516 more rows, and 15 more variables: dietpl_yr <dbl>, hcc <dbl>,
#   hcc_date <date>, hcc_yr <dbl>, b_lab_date <date>, b_alt <dbl>, b_bil <dbl>,
#   b_inr <dbl>, b_cr <dbl>, b_plt <dbl>, b_alb <dbl>, b_eag <dbl>,
#   b_eab <dbl>, b_dna <chr>, b_dna_log <dbl>
\end{verbatim}

\begin{itemize}
\tightlist
\item
  \textbf{복합 여러 조건 (AND + OR, not)}
\end{itemize}

\begin{Shaded}
\begin{Highlighting}[]
\CommentTok{\# 80세 이상의 남성이나(OR) 30세 미만의 여자}
\NormalTok{dat1 }\SpecialCharTok{\%\textgreater{}\%} 
  \FunctionTok{filter}\NormalTok{( (age}\SpecialCharTok{\textgreater{}}\DecValTok{80} \SpecialCharTok{\&}\NormalTok{ gender }\SpecialCharTok{==}\StringTok{"M"}\NormalTok{) }\SpecialCharTok{|}\NormalTok{ (age}\SpecialCharTok{\textless{}}\DecValTok{30} \SpecialCharTok{\&}\NormalTok{ gender }\SpecialCharTok{==}\StringTok{"F"}\NormalTok{)) }
\end{Highlighting}
\end{Shaded}

\begin{verbatim}
# A tibble: 20 x 23
   index_date gender   age last_date  treat_gr    lc dietpl dietpl_date
   <date>     <chr>  <dbl> <date>     <chr>    <dbl>  <dbl> <date>     
 1 2007-04-02 F         22 2010-05-25 ETV          0      0 2010-05-25 
 2 2007-07-03 F         24 2016-09-13 ETV          0      0 2016-09-13 
 3 2007-11-12 F         22 2009-08-17 ETV          0      0 2009-08-17 
 4 2008-02-04 F         29 2011-07-21 ETV          0      0 2011-07-21 
 5 2008-06-05 F         23 2017-09-29 ETV          1      0 2017-09-29 
 6 2008-07-11 F         25 2011-07-13 ETV          0      0 2011-07-13 
 7 2008-07-26 F         28 2017-02-20 ETV          1      0 2017-02-20 
 8 2008-12-23 F         28 2017-07-20 TDF          0      0 2017-07-20 
 9 2008-12-29 F         23 2017-09-11 TDF          0      0 2017-09-11 
10 2009-01-13 F         26 2017-06-02 TDF          0      0 2017-06-02 
11 2009-02-02 F         25 2013-05-23 TDF          0      0 2013-05-23 
12 2009-02-09 F         27 2014-06-09 TDF          0      0 2014-06-09 
13 2009-04-16 F         22 2013-01-09 TDF          0      0 2013-01-09 
14 2009-04-17 F         26 2015-01-12 TDF          0      0 2015-01-12 
15 2009-05-26 F         25 2013-02-18 TDF          0      0 2013-02-18 
16 2009-07-31 F         25 2012-12-10 TDF          0      0 2012-12-10 
17 2009-08-28 F         25 2017-03-24 TDF          1      0 2017-03-24 
18 2009-09-23 F         29 2017-06-21 TDF          0      0 2017-06-21 
19 2009-09-24 F         25 2013-01-10 TDF          0      0 2013-01-10 
20 2010-01-28 F         26 2012-03-27 TDF          1      0 2012-03-27 
# ... with 15 more variables: dietpl_yr <dbl>, hcc <dbl>, hcc_date <date>,
#   hcc_yr <dbl>, b_lab_date <date>, b_alt <dbl>, b_bil <dbl>, b_inr <dbl>,
#   b_cr <dbl>, b_plt <dbl>, b_alb <dbl>, b_eag <dbl>, b_eab <dbl>,
#   b_dna <chr>, b_dna_log <dbl>
\end{verbatim}

\begin{Shaded}
\begin{Highlighting}[]
\CommentTok{\# 80세 이상이면서 간경화가 없는 환자}
\NormalTok{dat1 }\SpecialCharTok{\%\textgreater{}\%} 
  \FunctionTok{filter}\NormalTok{(age }\SpecialCharTok{\textgreater{}=}\DecValTok{80} \SpecialCharTok{\&}\NormalTok{ lc}\SpecialCharTok{!=}\DecValTok{1}\NormalTok{)}
\end{Highlighting}
\end{Shaded}

\begin{verbatim}
# A tibble: 0 x 23
# ... with 23 variables: index_date <date>, gender <chr>, age <dbl>,
#   last_date <date>, treat_gr <chr>, lc <dbl>, dietpl <dbl>,
#   dietpl_date <date>, dietpl_yr <dbl>, hcc <dbl>, hcc_date <date>,
#   hcc_yr <dbl>, b_lab_date <date>, b_alt <dbl>, b_bil <dbl>, b_inr <dbl>,
#   b_cr <dbl>, b_plt <dbl>, b_alb <dbl>, b_eag <dbl>, b_eab <dbl>,
#   b_dna <chr>, b_dna_log <dbl>
\end{verbatim}

수학에서 먼저 해야되는 연산은 ()로 묶어 주듯이 조건별로 적절히 ()로
묶어준다.

\begin{itemize}
\tightlist
\item
  \textbf{유용한 명령어}
\end{itemize}

{[}index\_date{]}로 부터 연도(year)만 분리해서 variable을 새로 만들자

\begin{Shaded}
\begin{Highlighting}[]
\NormalTok{dat1}\SpecialCharTok{$}\NormalTok{year}\OtherTok{\textless{}{-}}\FunctionTok{year}\NormalTok{(dat1}\SpecialCharTok{$}\NormalTok{index\_date)}
\end{Highlighting}
\end{Shaded}

Error가 날겁니다.

\begin{Shaded}
\begin{Highlighting}[]
\FunctionTok{class}\NormalTok{(dat1}\SpecialCharTok{$}\NormalTok{index\_date)}
\end{Highlighting}
\end{Shaded}

\begin{verbatim}
[1] "Date"
\end{verbatim}

dat1\$index\_date의 형태는 ``Date'' 입니다.

``Date'' 형태의 data에서 자동으로 Year만 분리해줄 수 있는 함수가
필요합니다. (lubridate package)

\begin{Shaded}
\begin{Highlighting}[]
\FunctionTok{library}\NormalTok{(lubridate)}
\NormalTok{dat1}\SpecialCharTok{$}\NormalTok{year}\OtherTok{\textless{}{-}}\FunctionTok{year}\NormalTok{(dat1}\SpecialCharTok{$}\NormalTok{index\_date)}

\CommentTok{\# year 분포를 봅시다.}
\NormalTok{dat1 }\SpecialCharTok{\%\textgreater{}\%} 
   \FunctionTok{count}\NormalTok{(year)}
\end{Highlighting}
\end{Shaded}

\begin{verbatim}
# A tibble: 5 x 2
   year     n
* <dbl> <int>
1  2007   199
2  2008   315
3  2009   287
4  2010   193
5  2011     6
\end{verbatim}

만약 filter를 이용해서 2007, 2008, 2009년 자료만 골라낸다면?

\begin{Shaded}
\begin{Highlighting}[]
\NormalTok{dat1 }\SpecialCharTok{\%\textgreater{}\%} 
   \FunctionTok{filter}\NormalTok{(year }\SpecialCharTok{\textless{}=} \DecValTok{2009}\NormalTok{) }\CommentTok{\# 801명}
\end{Highlighting}
\end{Shaded}

\begin{verbatim}
# A tibble: 801 x 24
   index_date gender   age last_date  treat_gr    lc dietpl dietpl_date
   <date>     <chr>  <dbl> <date>     <chr>    <dbl>  <dbl> <date>     
 1 2007-01-05 M         54 2014-07-18 ETV          1      0 2017-09-13 
 2 2007-01-10 F         45 2016-08-25 ETV          0      0 2016-08-25 
 3 2007-01-11 M         49 2017-06-21 ETV          1      0 2017-06-21 
 4 2007-01-12 M         26 2012-12-17 ETV          0      0 2012-12-17 
 5 2007-01-18 M         50 2009-05-15 ETV          1      0 2009-05-15 
 6 2007-01-18 M         33 2013-01-11 ETV          1      0 2013-01-11 
 7 2007-01-26 M         49 2013-03-31 ETV          1      0 2017-08-07 
 8 2007-01-31 F         50 2017-08-24 ETV          0      0 2017-08-24 
 9 2007-01-31 F         49 2009-03-20 ETV          1      0 2009-03-20 
10 2007-02-01 M         50 2017-09-12 ETV          0      0 2017-09-12 
# ... with 791 more rows, and 16 more variables: dietpl_yr <dbl>, hcc <dbl>,
#   hcc_date <date>, hcc_yr <dbl>, b_lab_date <date>, b_alt <dbl>, b_bil <dbl>,
#   b_inr <dbl>, b_cr <dbl>, b_plt <dbl>, b_alb <dbl>, b_eag <dbl>,
#   b_eab <dbl>, b_dna <chr>, b_dna_log <dbl>, year <dbl>
\end{verbatim}

만약 2007, 2008, 2010년을 골라내야 한다면

\begin{Shaded}
\begin{Highlighting}[]
\NormalTok{dat1 }\SpecialCharTok{\%\textgreater{}\%} 
   \FunctionTok{filter}\NormalTok{(year}\SpecialCharTok{==}\DecValTok{2007} \SpecialCharTok{|}\NormalTok{ year}\SpecialCharTok{==}\DecValTok{2008} \SpecialCharTok{|}\NormalTok{ year}\SpecialCharTok{==}\DecValTok{2010}\NormalTok{) }\CommentTok{\# 707명}
\end{Highlighting}
\end{Shaded}

\begin{verbatim}
# A tibble: 707 x 24
   index_date gender   age last_date  treat_gr    lc dietpl dietpl_date
   <date>     <chr>  <dbl> <date>     <chr>    <dbl>  <dbl> <date>     
 1 2007-01-05 M         54 2014-07-18 ETV          1      0 2017-09-13 
 2 2007-01-10 F         45 2016-08-25 ETV          0      0 2016-08-25 
 3 2007-01-11 M         49 2017-06-21 ETV          1      0 2017-06-21 
 4 2007-01-12 M         26 2012-12-17 ETV          0      0 2012-12-17 
 5 2007-01-18 M         50 2009-05-15 ETV          1      0 2009-05-15 
 6 2007-01-18 M         33 2013-01-11 ETV          1      0 2013-01-11 
 7 2007-01-26 M         49 2013-03-31 ETV          1      0 2017-08-07 
 8 2007-01-31 F         50 2017-08-24 ETV          0      0 2017-08-24 
 9 2007-01-31 F         49 2009-03-20 ETV          1      0 2009-03-20 
10 2007-02-01 M         50 2017-09-12 ETV          0      0 2017-09-12 
# ... with 697 more rows, and 16 more variables: dietpl_yr <dbl>, hcc <dbl>,
#   hcc_date <date>, hcc_yr <dbl>, b_lab_date <date>, b_alt <dbl>, b_bil <dbl>,
#   b_inr <dbl>, b_cr <dbl>, b_plt <dbl>, b_alb <dbl>, b_eag <dbl>,
#   b_eab <dbl>, b_dna <chr>, b_dna_log <dbl>, year <dbl>
\end{verbatim}

하지만 골라내야될 조건이 많아지거나 불규칙적이다면?

\%in\%을 사용해보자

\begin{Shaded}
\begin{Highlighting}[]
\NormalTok{dat1 }\SpecialCharTok{\%\textgreater{}\%} 
   \FunctionTok{filter}\NormalTok{(year }\SpecialCharTok{\%in\%} \FunctionTok{c}\NormalTok{(}\DecValTok{2007}\NormalTok{, }\DecValTok{2008}\NormalTok{, }\DecValTok{2010}\NormalTok{)) }\CommentTok{\# 같은 결과}
\end{Highlighting}
\end{Shaded}

\begin{verbatim}
# A tibble: 707 x 24
   index_date gender   age last_date  treat_gr    lc dietpl dietpl_date
   <date>     <chr>  <dbl> <date>     <chr>    <dbl>  <dbl> <date>     
 1 2007-01-05 M         54 2014-07-18 ETV          1      0 2017-09-13 
 2 2007-01-10 F         45 2016-08-25 ETV          0      0 2016-08-25 
 3 2007-01-11 M         49 2017-06-21 ETV          1      0 2017-06-21 
 4 2007-01-12 M         26 2012-12-17 ETV          0      0 2012-12-17 
 5 2007-01-18 M         50 2009-05-15 ETV          1      0 2009-05-15 
 6 2007-01-18 M         33 2013-01-11 ETV          1      0 2013-01-11 
 7 2007-01-26 M         49 2013-03-31 ETV          1      0 2017-08-07 
 8 2007-01-31 F         50 2017-08-24 ETV          0      0 2017-08-24 
 9 2007-01-31 F         49 2009-03-20 ETV          1      0 2009-03-20 
10 2007-02-01 M         50 2017-09-12 ETV          0      0 2017-09-12 
# ... with 697 more rows, and 16 more variables: dietpl_yr <dbl>, hcc <dbl>,
#   hcc_date <date>, hcc_yr <dbl>, b_lab_date <date>, b_alt <dbl>, b_bil <dbl>,
#   b_inr <dbl>, b_cr <dbl>, b_plt <dbl>, b_alb <dbl>, b_eag <dbl>,
#   b_eab <dbl>, b_dna <chr>, b_dna_log <dbl>, year <dbl>
\end{verbatim}

between을 이용해도 된다

\begin{Shaded}
\begin{Highlighting}[]
\CommentTok{\# 30세이상 80세 이하}
\NormalTok{dat1 }\SpecialCharTok{\%\textgreater{}\%} 
   \FunctionTok{filter}\NormalTok{(age }\SpecialCharTok{\textgreater{}=}\DecValTok{30} \SpecialCharTok{\&}\NormalTok{ age }\SpecialCharTok{\textless{}}\DecValTok{80}\NormalTok{) }\CommentTok{\# 926명}
\end{Highlighting}
\end{Shaded}

\begin{verbatim}
# A tibble: 926 x 24
   index_date gender   age last_date  treat_gr    lc dietpl dietpl_date
   <date>     <chr>  <dbl> <date>     <chr>    <dbl>  <dbl> <date>     
 1 2007-01-05 M         54 2014-07-18 ETV          1      0 2017-09-13 
 2 2007-01-10 F         45 2016-08-25 ETV          0      0 2016-08-25 
 3 2007-01-11 M         49 2017-06-21 ETV          1      0 2017-06-21 
 4 2007-01-18 M         50 2009-05-15 ETV          1      0 2009-05-15 
 5 2007-01-18 M         33 2013-01-11 ETV          1      0 2013-01-11 
 6 2007-01-26 M         49 2013-03-31 ETV          1      0 2017-08-07 
 7 2007-01-31 F         50 2017-08-24 ETV          0      0 2017-08-24 
 8 2007-01-31 F         49 2009-03-20 ETV          1      0 2009-03-20 
 9 2007-02-01 M         50 2017-09-12 ETV          0      0 2017-09-12 
10 2007-02-01 M         32 2010-09-30 ETV          0      0 2010-09-30 
# ... with 916 more rows, and 16 more variables: dietpl_yr <dbl>, hcc <dbl>,
#   hcc_date <date>, hcc_yr <dbl>, b_lab_date <date>, b_alt <dbl>, b_bil <dbl>,
#   b_inr <dbl>, b_cr <dbl>, b_plt <dbl>, b_alb <dbl>, b_eag <dbl>,
#   b_eab <dbl>, b_dna <chr>, b_dna_log <dbl>, year <dbl>
\end{verbatim}

\begin{Shaded}
\begin{Highlighting}[]
\CommentTok{\# between 이용시}
\NormalTok{dat1 }\SpecialCharTok{\%\textgreater{}\%} 
   \FunctionTok{filter}\NormalTok{(}\FunctionTok{between}\NormalTok{(age, }\DecValTok{30}\NormalTok{,}\DecValTok{80}\NormalTok{)) }\CommentTok{\#이상, 이하의미로 30, 80은 포함됨!}
\end{Highlighting}
\end{Shaded}

\begin{verbatim}
# A tibble: 926 x 24
   index_date gender   age last_date  treat_gr    lc dietpl dietpl_date
   <date>     <chr>  <dbl> <date>     <chr>    <dbl>  <dbl> <date>     
 1 2007-01-05 M         54 2014-07-18 ETV          1      0 2017-09-13 
 2 2007-01-10 F         45 2016-08-25 ETV          0      0 2016-08-25 
 3 2007-01-11 M         49 2017-06-21 ETV          1      0 2017-06-21 
 4 2007-01-18 M         50 2009-05-15 ETV          1      0 2009-05-15 
 5 2007-01-18 M         33 2013-01-11 ETV          1      0 2013-01-11 
 6 2007-01-26 M         49 2013-03-31 ETV          1      0 2017-08-07 
 7 2007-01-31 F         50 2017-08-24 ETV          0      0 2017-08-24 
 8 2007-01-31 F         49 2009-03-20 ETV          1      0 2009-03-20 
 9 2007-02-01 M         50 2017-09-12 ETV          0      0 2017-09-12 
10 2007-02-01 M         32 2010-09-30 ETV          0      0 2010-09-30 
# ... with 916 more rows, and 16 more variables: dietpl_yr <dbl>, hcc <dbl>,
#   hcc_date <date>, hcc_yr <dbl>, b_lab_date <date>, b_alt <dbl>, b_bil <dbl>,
#   b_inr <dbl>, b_cr <dbl>, b_plt <dbl>, b_alb <dbl>, b_eag <dbl>,
#   b_eab <dbl>, b_dna <chr>, b_dna_log <dbl>, year <dbl>
\end{verbatim}

주의! between 이용시 지정값은 포함됨

\begin{itemize}
\tightlist
\item
  \textbf{결측값이 없는 자료만 filter}
\end{itemize}

\begin{Shaded}
\begin{Highlighting}[]
\CommentTok{\# b\_inr (prothrombine time INR)의 결측값은 몇개?}
\FunctionTok{summary}\NormalTok{(dat1}\SpecialCharTok{$}\NormalTok{b\_inr)}
\end{Highlighting}
\end{Shaded}

\begin{verbatim}
   Min. 1st Qu.  Median    Mean 3rd Qu.    Max.    NA's 
  0.840   1.030   1.090   1.134   1.180   3.130      16 
\end{verbatim}

\begin{Shaded}
\begin{Highlighting}[]
\CommentTok{\# 혹은}
\FunctionTok{sum}\NormalTok{(}\FunctionTok{is.na}\NormalTok{(dat1}\SpecialCharTok{$}\NormalTok{b\_inr)) }
\end{Highlighting}
\end{Shaded}

\begin{verbatim}
[1] 16
\end{verbatim}

HBeAg 값이 있는 자료만 filter 해보자

\begin{Shaded}
\begin{Highlighting}[]
\NormalTok{dat1 }\SpecialCharTok{\%\textgreater{}\%} 
   \FunctionTok{filter}\NormalTok{( }\FunctionTok{is.na}\NormalTok{(b\_inr)) }\CommentTok{\# 이렇게 하면 b\_inr이 NA만 나온다}
\end{Highlighting}
\end{Shaded}

\begin{verbatim}
# A tibble: 16 x 24
   index_date gender   age last_date  treat_gr    lc dietpl dietpl_date
   <date>     <chr>  <dbl> <date>     <chr>    <dbl>  <dbl> <date>     
 1 2007-01-31 F         50 2017-08-24 ETV          0      0 2017-08-24 
 2 2007-02-15 M         52 2011-10-13 ETV          1      0 2011-10-13 
 3 2007-08-01 M         58 2013-03-31 ETV          1      0 2017-09-20 
 4 2007-08-01 M         48 2009-09-30 ETV          1      0 2009-09-30 
 5 2007-08-02 M         48 2017-09-14 ETV          1      0 2017-09-14 
 6 2007-08-02 M         48 2017-07-04 ETV          1      0 2017-07-04 
 7 2007-08-07 M         25 2009-11-05 ETV          0      0 2009-11-05 
 8 2007-08-09 M         48 2009-08-27 ETV          0      0 2009-08-27 
 9 2008-06-24 M         46 2013-07-16 ETV          1      0 2013-07-16 
10 2008-06-24 M         41 2012-12-28 ETV          0      0 2012-12-28 
11 2008-06-26 F         53 2017-09-13 ETV          1      0 2017-09-13 
12 2008-06-26 F         48 2011-09-05 ETV          1      0 2014-07-15 
13 2008-06-26 M         45 2017-07-31 ETV          1      0 2017-07-31 
14 2008-06-26 F         52 2017-06-14 ETV          0      0 2017-06-14 
15 2008-06-27 F         30 2013-04-16 ETV          0      0 2013-04-16 
16 2008-06-27 M         44 2017-07-04 ETV          0      0 2017-07-04 
# ... with 16 more variables: dietpl_yr <dbl>, hcc <dbl>, hcc_date <date>,
#   hcc_yr <dbl>, b_lab_date <date>, b_alt <dbl>, b_bil <dbl>, b_inr <dbl>,
#   b_cr <dbl>, b_plt <dbl>, b_alb <dbl>, b_eag <dbl>, b_eab <dbl>,
#   b_dna <chr>, b_dna_log <dbl>, year <dbl>
\end{verbatim}

\begin{Shaded}
\begin{Highlighting}[]
\CommentTok{\# 따라서}
\NormalTok{dat1 }\SpecialCharTok{\%\textgreater{}\%} 
   \FunctionTok{filter}\NormalTok{( }\SpecialCharTok{!}\FunctionTok{is.na}\NormalTok{(b\_inr)) }\CommentTok{\#즉 NA가 !(not)값들만 나온다.}
\end{Highlighting}
\end{Shaded}

\begin{verbatim}
# A tibble: 984 x 24
   index_date gender   age last_date  treat_gr    lc dietpl dietpl_date
   <date>     <chr>  <dbl> <date>     <chr>    <dbl>  <dbl> <date>     
 1 2007-01-05 M         54 2014-07-18 ETV          1      0 2017-09-13 
 2 2007-01-10 F         45 2016-08-25 ETV          0      0 2016-08-25 
 3 2007-01-11 M         49 2017-06-21 ETV          1      0 2017-06-21 
 4 2007-01-12 M         26 2012-12-17 ETV          0      0 2012-12-17 
 5 2007-01-18 M         50 2009-05-15 ETV          1      0 2009-05-15 
 6 2007-01-18 M         33 2013-01-11 ETV          1      0 2013-01-11 
 7 2007-01-26 M         49 2013-03-31 ETV          1      0 2017-08-07 
 8 2007-01-31 F         49 2009-03-20 ETV          1      0 2009-03-20 
 9 2007-02-01 M         50 2017-09-12 ETV          0      0 2017-09-12 
10 2007-02-01 M         32 2010-09-30 ETV          0      0 2010-09-30 
# ... with 974 more rows, and 16 more variables: dietpl_yr <dbl>, hcc <dbl>,
#   hcc_date <date>, hcc_yr <dbl>, b_lab_date <date>, b_alt <dbl>, b_bil <dbl>,
#   b_inr <dbl>, b_cr <dbl>, b_plt <dbl>, b_alb <dbl>, b_eag <dbl>,
#   b_eab <dbl>, b_dna <chr>, b_dna_log <dbl>, year <dbl>
\end{verbatim}

\end{document}
